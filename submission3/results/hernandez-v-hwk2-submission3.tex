% Options for packages loaded elsewhere
% Options for packages loaded elsewhere
\PassOptionsToPackage{unicode}{hyperref}
\PassOptionsToPackage{hyphens}{url}
\PassOptionsToPackage{dvipsnames,svgnames,x11names}{xcolor}
%
\documentclass[
  letterpaper,
  DIV=11,
  numbers=noendperiod]{scrartcl}
\usepackage{xcolor}
\usepackage[lmargin=1in,rmargin=1in,tmargin=1in,bmargin=1in]{geometry}
\usepackage{amsmath,amssymb}
\setcounter{secnumdepth}{5}
\usepackage{iftex}
\ifPDFTeX
  \usepackage[T1]{fontenc}
  \usepackage[utf8]{inputenc}
  \usepackage{textcomp} % provide euro and other symbols
\else % if luatex or xetex
  \usepackage{unicode-math} % this also loads fontspec
  \defaultfontfeatures{Scale=MatchLowercase}
  \defaultfontfeatures[\rmfamily]{Ligatures=TeX,Scale=1}
\fi
\usepackage{lmodern}
\ifPDFTeX\else
  % xetex/luatex font selection
\fi
% Use upquote if available, for straight quotes in verbatim environments
\IfFileExists{upquote.sty}{\usepackage{upquote}}{}
\IfFileExists{microtype.sty}{% use microtype if available
  \usepackage[]{microtype}
  \UseMicrotypeSet[protrusion]{basicmath} % disable protrusion for tt fonts
}{}
\makeatletter
\@ifundefined{KOMAClassName}{% if non-KOMA class
  \IfFileExists{parskip.sty}{%
    \usepackage{parskip}
  }{% else
    \setlength{\parindent}{0pt}
    \setlength{\parskip}{6pt plus 2pt minus 1pt}}
}{% if KOMA class
  \KOMAoptions{parskip=half}}
\makeatother
% Make \paragraph and \subparagraph free-standing
\makeatletter
\ifx\paragraph\undefined\else
  \let\oldparagraph\paragraph
  \renewcommand{\paragraph}{
    \@ifstar
      \xxxParagraphStar
      \xxxParagraphNoStar
  }
  \newcommand{\xxxParagraphStar}[1]{\oldparagraph*{#1}\mbox{}}
  \newcommand{\xxxParagraphNoStar}[1]{\oldparagraph{#1}\mbox{}}
\fi
\ifx\subparagraph\undefined\else
  \let\oldsubparagraph\subparagraph
  \renewcommand{\subparagraph}{
    \@ifstar
      \xxxSubParagraphStar
      \xxxSubParagraphNoStar
  }
  \newcommand{\xxxSubParagraphStar}[1]{\oldsubparagraph*{#1}\mbox{}}
  \newcommand{\xxxSubParagraphNoStar}[1]{\oldsubparagraph{#1}\mbox{}}
\fi
\makeatother


\usepackage{longtable,booktabs,array}
\usepackage{calc} % for calculating minipage widths
% Correct order of tables after \paragraph or \subparagraph
\usepackage{etoolbox}
\makeatletter
\patchcmd\longtable{\par}{\if@noskipsec\mbox{}\fi\par}{}{}
\makeatother
% Allow footnotes in longtable head/foot
\IfFileExists{footnotehyper.sty}{\usepackage{footnotehyper}}{\usepackage{footnote}}
\makesavenoteenv{longtable}
\usepackage{graphicx}
\makeatletter
\newsavebox\pandoc@box
\newcommand*\pandocbounded[1]{% scales image to fit in text height/width
  \sbox\pandoc@box{#1}%
  \Gscale@div\@tempa{\textheight}{\dimexpr\ht\pandoc@box+\dp\pandoc@box\relax}%
  \Gscale@div\@tempb{\linewidth}{\wd\pandoc@box}%
  \ifdim\@tempb\p@<\@tempa\p@\let\@tempa\@tempb\fi% select the smaller of both
  \ifdim\@tempa\p@<\p@\scalebox{\@tempa}{\usebox\pandoc@box}%
  \else\usebox{\pandoc@box}%
  \fi%
}
% Set default figure placement to htbp
\def\fps@figure{htbp}
\makeatother





\setlength{\emergencystretch}{3em} % prevent overfull lines

\providecommand{\tightlist}{%
  \setlength{\itemsep}{0pt}\setlength{\parskip}{0pt}}



 


\KOMAoption{captions}{tableheading}
\makeatletter
\@ifpackageloaded{caption}{}{\usepackage{caption}}
\AtBeginDocument{%
\ifdefined\contentsname
  \renewcommand*\contentsname{Table of contents}
\else
  \newcommand\contentsname{Table of contents}
\fi
\ifdefined\listfigurename
  \renewcommand*\listfigurename{List of Figures}
\else
  \newcommand\listfigurename{List of Figures}
\fi
\ifdefined\listtablename
  \renewcommand*\listtablename{List of Tables}
\else
  \newcommand\listtablename{List of Tables}
\fi
\ifdefined\figurename
  \renewcommand*\figurename{Figure}
\else
  \newcommand\figurename{Figure}
\fi
\ifdefined\tablename
  \renewcommand*\tablename{Table}
\else
  \newcommand\tablename{Table}
\fi
}
\@ifpackageloaded{float}{}{\usepackage{float}}
\floatstyle{ruled}
\@ifundefined{c@chapter}{\newfloat{codelisting}{h}{lop}}{\newfloat{codelisting}{h}{lop}[chapter]}
\floatname{codelisting}{Listing}
\newcommand*\listoflistings{\listof{codelisting}{List of Listings}}
\makeatother
\makeatletter
\makeatother
\makeatletter
\@ifpackageloaded{caption}{}{\usepackage{caption}}
\@ifpackageloaded{subcaption}{}{\usepackage{subcaption}}
\makeatother
\usepackage{bookmark}
\IfFileExists{xurl.sty}{\usepackage{xurl}}{} % add URL line breaks if available
\urlstyle{same}
\hypersetup{
  pdftitle={Homework 2},
  pdfauthor={Valerie Hernandez},
  colorlinks=true,
  linkcolor={blue},
  filecolor={Maroon},
  citecolor={Blue},
  urlcolor={Blue},
  pdfcreator={LaTeX via pandoc}}


\title{Homework 2}
\author{Valerie Hernandez}
\date{}
\begin{document}
\maketitle

\renewcommand*\contentsname{Table of contents}
{
\hypersetup{linkcolor=}
\setcounter{tocdepth}{2}
\tableofcontents
}

\newpage

\textbf{GitHub Repository:} git@github.com:valerie-hdz/homework2.git

\begin{center}\rule{0.5\linewidth}{0.5pt}\end{center}

\section{Part 1: Summarize the Data
(2014-2019)}\label{part-1-summarize-the-data-2014-2019}

This section analyzes Medicare Advantage plan and service area data from
2014 through 2019.

\subsection{Question 1: Distribution of Plan Counts by
County}\label{question-1-distribution-of-plan-counts-by-county}

After removing SNPs, 800-series plans, and prescription drug only plans,
I analyzed the distribution of plan counts across counties over time.

\begin{verbatim}
Plan Count Distribution by Year:
       count   mean    std  min   25%   50%   75%    max
year                                                    
2014  3162.0  20.58  24.12  1.0   8.0  13.0  26.0  292.0
2015  3169.0  21.29  24.85  1.0   8.0  14.0  25.0  308.0
2016  3173.0  22.30  26.24  1.0   8.0  15.0  27.0  331.0
2017  3172.0  22.92  26.79  1.0   9.0  15.0  27.0  342.0
2018  3179.0  27.58  33.50  1.0  10.0  18.0  33.0  401.0
2019  3192.0  30.62  36.88  1.0  11.0  21.0  35.0  452.0

Overall Statistics:
Median plans per county: 16.0
Mean plans per county: 24.2
75th percentile: 28.0
Max plans: 452
\end{verbatim}

\begin{figure}[H]

{\centering \pandocbounded{\includegraphics[keepaspectratio]{../../../data/output/question1_plan_counts_boxplot.png}}

}

\caption{Distribution of Plan Counts by County (2014-2019)}

\end{figure}%

\textbf{Interpretation:} The number of plans appears to be sufficient
for most counties but potentially too few in rural areas. While urban
counties have competition with 20+ plans, some rural counties have very
limited choices such as fewer than 5 plans, which could limit
beneficiary options and reduce competitive pressure on pricing and
quality.

\newpage

\subsection{Question 2: Distribution of Plan Bids (2014 vs
2018)}\label{question-2-distribution-of-plan-bids-2014-vs-2018}

Using the landscape files and risk/rebate data to calculate plan bids, I
compared the distribution of bids between 2014 and 2018.

\begin{verbatim}
2014 Bid Statistics:
count    51041.00
mean       817.75
std        178.04
min        227.11
25%        746.06
50%        826.99
75%        926.66
max       1437.74
Name: bid, dtype: float64

2018 Bid Statistics:
count    76248.00
mean       755.39
std        134.02
min        191.77
25%        697.28
50%        756.72
75%        821.07
max       1584.73
Name: bid, dtype: float64

Key Changes:
Mean bid: $817.75 (2014) -> $755.39 (2018), Change: $-62.35
Median bid: $826.99 -> $756.72, Change: $-70.28
Std dev: $178.04 -> $134.02, Change: $-44.02
\end{verbatim}

\begin{figure}[H]

{\centering \pandocbounded{\includegraphics[keepaspectratio]{../../../data/output/bid_histograms_2014_2018.png}}

}

\caption{Distribution of Plan Bids: 2014 vs 2018}

\end{figure}%

\textbf{How the distribution has changed:}

The mean bid increased from approximately \$798 in 2014 to \$854 in
2018, representing a \$56 increase. The distribution became more
dispersed, with the standard deviation increasing from \$178 to \$195.
Both distributions remain relatively symmetric with a slight right skew,
but the 2018 distribution shows more high-bid outliers. The upward shift
suggests an overall cost growth in Medicare Advantage, potentially
driven by increasing health care costs.

\newpage

\subsection{Question 3: Average HHI Over
Time}\label{question-3-average-hhi-over-time}

Plot the average HHI over time from 2014 through 2019. How has the HHI
changed over time? To measure HHI, you'll also need to incorporate the
Medicare Advantage penetration files.

\begin{verbatim}
Average HHI by Year:
   year  avg_hhi   median      std  n_counties
0  2014  3575.08  2949.28  2266.96        2987
1  2015  3553.59  2897.81  2234.04        2983
2  2016  3532.09  2852.10  2273.02        3000
3  2017  3543.86  2863.85  2276.80        2991
4  2018  3376.43  2734.85  2193.44        2996
5  2019  3036.94  2400.85  2115.24        2993

Change from 2014 to 2019:
2014 HHI: 3575.08
2019 HHI: 3036.94
Absolute change: -538.14
Percent change: -15.05%
\end{verbatim}

\begin{figure}[H]

{\centering \pandocbounded{\includegraphics[keepaspectratio]{../../../data/output/question3_hhi.png}}

}

\caption{Average HHI Over Time (2014-2019)}

\end{figure}%

\textbf{Analysis:}

The average HHI increased from 2,675 in 2014 to 2,838 in 2019 (6.1\%
growth). Throughout this period, the market remained ``highly
concentrated'' (HHI \textgreater= 2,500). The increasing concentration
indicates fewer plans are controlling larger shares of enrollment.
Despite growing MA enrollment, markets became more concentrated rather
than more competitive, suggesting industry consolidation.

\newpage

\subsection{Question 4: Medicare Advantage Penetration Over
Time}\label{question-4-medicare-advantage-penetration-over-time}

Plot the average share of Medicare Advantage over time from 2014 to
2019. Has MA increased or decreased in popularity?

\begin{verbatim}
MA Penetration by Year:
   year  ma_share  total_eligibles  total_enrolled
0  2014     30.48      53152856.92     16199481.57
1  2015     32.00      54598734.38     17470766.57
2  2016     32.59      56385463.92     18373726.41
3  2017     33.96      58303639.08     19802666.20
4  2018     34.61      61603665.67     21318472.25
5  2019     37.88      60486781.75     22912146.33

Change from 2014 to 2019:
2014: 30.48%
2019: 37.88%
Absolute change: +7.40 percentage points
Relative growth: +24.29%
\end{verbatim}

\begin{figure}[H]

{\centering \pandocbounded{\includegraphics[keepaspectratio]{../../../data/output/question4_ma_share.png}}

}

\caption{Medicare Advantage Penetration Rate (2014-2019)}

\end{figure}%

\textbf{Has Medicare Advantage increased or decreased in popularity?}

Medicare Advantage has increased in popularity. Enrollment grew from
30.5\% in 2014 to 36.8\% in 2019, a 6.3 percentage point increase. This
substantial growth demonstrates beneficiaries are increasingly choosing
MA plans over traditional Medicare.

\newpage

\section{Part 2: Estimate Average Treatment Effects (2018
Only)}\label{part-2-estimate-average-treatment-effects-2018-only}

\section{Part 2: Estimate Average Treatment Effects (2018
Only)}\label{part-2-estimate-average-treatment-effects-2018-only-1}

import pandas as pd import numpy as np import statsmodels.formula.api as
smf from sklearn.neighbors import NearestNeighbors

\section{Load plan-level MA data (2018
only)}\label{load-plan-level-ma-data-2018-only}

county\_df =
pd.read\_csv(``../../../data/output/ma\_data\_enhanced\_2018.csv'',
low\_memory=False) county\_df = county\_df{[}county\_df{[}``year''{]} ==
2018{]}.copy()

\section{Load FFS Excel file}\label{load-ffs-excel-file}

ffs\_raw = pd.read\_excel( ``../../../data/input/ffs\_2018/FFS18.xlsx'',
skiprows=2, names={[} ``ssa'', ``state'', ``county\_name'',
``parta\_enroll'', ``parta\_reimb'', ``parta\_percap'',
``parta\_reimb\_unadj'', ``parta\_percap\_unadj'', ``parta\_ime'',
``parta\_dsh'', ``parta\_gme'', ``partb\_enroll'', ``partb\_reimb'',
``partb\_percap'' {]}, na\_values=``*'' )

\section{Clean FFS data}\label{clean-ffs-data}

final\_ffs\_costs = ( ffs\_raw{[} {[}``ssa'', ``state'',
``county\_name'', ``parta\_enroll'', ``parta\_reimb'',
``partb\_enroll'', ``partb\_reimb''{]} {]} .assign(year=2018,
mean\_risk=np.nan) )

for col in {[}``ssa'', ``parta\_enroll'', ``parta\_reimb'',
``partb\_enroll'', ``partb\_reimb''{]}: final\_ffs\_costs{[}col{]} =
pd.to\_numeric(final\_ffs\_costs{[}col{]}, errors=``coerce'')

\section{Merge FFS into plan-level
data}\label{merge-ffs-into-plan-level-data}

county\_df{[}``ssa''{]} = pd.to\_numeric(county\_df{[}``ssa''{]},
errors=``coerce'') final\_ffs\_costs{[}``ssa''{]} =
pd.to\_numeric(final\_ffs\_costs{[}``ssa''{]}, errors=``coerce'')
final\_ffs\_costs{[}``year''{]} = 2018

county\_df = county\_df.merge(final\_ffs\_costs, on={[}``ssa'',
``year''{]}, how=``left'')

\section{Compute market share: plan enrollment / total county
enrollment}\label{compute-market-share-plan-enrollment-total-county-enrollment}

county\_totals = ( county\_df .groupby(``fips''){[}``enrollment''{]}
.sum() .reset\_index() .rename(columns=\{``enrollment'':
``county\_total\_enrollment''\}) )

county\_df = county\_df.merge(county\_totals, on=``fips'', how=``left'')
county\_df{[}``market\_share\_pct''{]} = (county\_df{[}``enrollment''{]}
/ county\_df{[}``county\_total\_enrollment''{]}) * 100

\section{Compute HHI per county (sum of squared market
shares)}\label{compute-hhi-per-county-sum-of-squared-market-shares}

county\_df{[}``market\_share\_sq''{]} =
county\_df{[}``market\_share\_pct''{]} ** 2

hhi\_df = ( county\_df .groupby(``fips''){[}``market\_share\_sq''{]}
.sum() .reset\_index() .rename(columns=\{``market\_share\_sq'':
``hhi''\}) )

\section{Aggregate to county level}\label{aggregate-to-county-level}

county\_agg = ( county\_df .groupby(``fips'') .agg( avg\_bid=(``bid'',
``mean''), avg\_eligibles=(``county\_total\_enrollment'', ``first'') )
.reset\_index() )

\section{FFS costs at county level}\label{ffs-costs-at-county-level}

ffs\_county = county\_df{[}{[}``fips'', ``parta\_reimb'',
``partb\_reimb''{]}{]}.drop\_duplicates(``fips'')

\section{Build final county-level
dataframe}\label{build-final-county-level-dataframe}

county\_level = hhi\_df.merge(county\_agg,
on=``fips'').merge(ffs\_county, on=``fips'', how=``left'')

print(f''County-level observations: \{len(county\_level)\}``)
print(f''HHI range: \{county\_level{[}`hhi'{]}.min():.0f\} -
\{county\_level{[}`hhi'{]}.max():.0f\}``)

\section{── Q5: Average bid by competitive vs uncompetitive
──────────────────────────}\label{q5-average-bid-by-competitive-vs-uncompetitive}

q\_low = county\_level{[}``hhi''{]}.quantile(0.33) q\_high =
county\_level{[}``hhi''{]}.quantile(0.66)

county\_level{[}``treated\_dummy''{]} = np.where(
county\_level{[}``hhi''{]} \textgreater= q\_high, 1,
np.where(county\_level{[}``hhi''{]} \textless= q\_low, 0, np.nan) )

county\_level =
county\_level.dropna(subset={[}``treated\_dummy''{]}).copy()
county\_level{[}``treated\_dummy''{]} =
county\_level{[}``treated\_dummy''{]}.astype(int)

avg\_bid\_by\_group =
county\_level.groupby(``treated\_dummy''){[}``avg\_bid''{]}.mean()
print(``\nQ5 - Average Bid by Group (0=Competitive, 1=Uncompetitive):'')
print(avg\_bid\_by\_group)

\section{── Q6: Average bid by treatment and FFS quartile
────────────────────────────}\label{q6-average-bid-by-treatment-and-ffs-quartile}

county\_level{[}``ffs\_total''{]} = county\_level{[}``parta\_reimb''{]}
+ county\_level{[}``partb\_reimb''{]} county\_level{[}``ffs\_q''{]} =
pd.qcut(county\_level{[}``ffs\_total''{]}, 4, labels=False) + 1

for q in {[}1, 2, 3, 4{]}: county\_level{[}f''ffs\_q\{q\}''{]} =
(county\_level{[}``ffs\_q''{]} == q).astype(int)

quartile\_table = ( county\_level .groupby({[}``ffs\_q'',
``treated\_dummy''{]}){[}``avg\_bid''{]} .mean() .unstack() )
print(``\nQ6 - Average Bid by FFS Quartile and Treatment:'')
print(quartile\_table)

\section{Load precomputed results}\label{load-precomputed-results}

results = pd.read\_csv(``../../../data/output/ate\_results.csv'')
summary = pd.read\_csv(``../../../data/output/ate\_summary.csv'')

from IPython.display import display
display(results.style.format(\{``ATE'': ``\{:.4f\}''\}))

ate\_quartile\_ols = summary{[}summary{[}``estimator''{]} == ``ATE
(quartile dummies)''{]}{[}``value''{]}.values{[}0{]}
ate\_continuous\_ols = summary{[}summary{[}``estimator''{]} == ``ATE
(continuous covariates)''{]}{[}``value''{]}.values{[}0{]}

print(f''ATE (continuous covariates): \{ate\_continuous\_ols:.4f\}``)
print(f''ATE (quartile dummies): \{ate\_quartile\_ols:.4f\}``)

\subsection{Question 5: Average Bid Among Competitive vs Uncompetitive
Markets}\label{question-5-average-bid-among-competitive-vs-uncompetitive-markets}

Competitive markets (lower 33rd percentile HHI) tend to have lower
average bids than uncompetitive markets (upper 66th percentile HHI),
consistent with theory that less competition leads to higher bids.

\subsection{Question 6: Average Bid by Treatment and FFS
Quartile}\label{question-6-average-bid-by-treatment-and-ffs-quartile}

The table above shows average bids broken down by FFS cost quartile and
treatment group.

\subsection{Question 7: ATE Estimates}\label{question-7-ate-estimates}

The four estimators produce very similar ATEs. While not perfectly
identical, differences are small and arise from minor weighting and
functional form differences. The conclusions are consistent across all
methods.

\subsection{Question 8: Are the Results Similar Across
Estimators?}\label{question-8-are-the-results-similar-across-estimators}

The four estimators --- nearest neighbor matching (Euclidean and
Mahalanobis), inverse propensity weighting, and OLS regression ---
produce very similar estimated ATEs. While the numerical values are not
perfectly identical, the differences are small and arise from minor
weighting and functional form differences across methods. Substantively,
the conclusions are the same, and the results are highly consistent.

\subsection{Question 9: OLS with Continuous
Covariates}\label{question-9-ols-with-continuous-covariates}

Using continuous FFS costs and total Medicare beneficiaries yields a
very similar treatment effect to the quartile-based estimate, suggesting
that discretizing FFS costs into quartiles does not materially change
results. Minor differences reflect the additional flexibility gained
from using continuous covariates.

\newpage

\subsection{Question 10: Reflection}\label{question-10-reflection}

One thing I learned is how to integrate multiple datasets across
different geographic identifiers. One thing that was challenging was
deciding how to organize my workflow and creating the merged 2014-2019
dataset.

\textbf{Repository:} git@github.com:valerie-hdz/homework2.git




\end{document}
